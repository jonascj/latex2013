\documentclass[12pt, a4paper]{article}

\usepackage[utf8]{inputenc}
\usepackage[T1]{fontenc}
\usepackage{hyperref}

\usepackage[top=2cm, bottom=3cm, left=2.5cm, right=2.5cm]{geometry}
\usepackage[danish]{babel}


\title{LaTeX Minikursus\\Ugeseddel 0}
\author{
	Allan Grønhøj Hansen\thanks{\href{mailto:alhan08@student.sdu.dk}{\nolinkurl{alhan08@student.sdu.dk}}}{ }
	og Jonas Camillus Jeppesen\thanks{\href{mailto:jojep07@student.sdu.dk}{\nolinkurl{jojep07@student.sdu.dk}}}
}
\date{\today}

\begin{document}
\maketitle
\thispagestyle{empty}

\section{Velkommen til LaTeX-minikurset}
Minikurset består af tre undervisningsgange (á 3 timer). Hver undervisningsgang vil starte med en kort instroduktion til dagens LaTeX emner hvorefter I bliver sat i gang med at lave opgaver/eksempler. På minikursets BlackBoard side kan I finde en undervisningsbog som vil blive benyttet som reference i løbet af kurset.

\section{Installation af LaTeX før første undervisningsgang}
Da undervisningen i høj grad består af jer som skal lave eksempler/opgaver er det vigtigt, at I har installeret/opsat LaTeX på jeres egen bærbare computer (og, at I medbringer denne bærbare computer, dooh). På BlackBoard-siden for LaTeX-minikurset ligger der et dokument "Installationsguide til LaTeX"{ }som forklarer hvordan I installerer/opsætter LaTeX i MacOS, Ubuntu og Windows. Vi kan ikke (og vil heller ikke) bruge tid på at installere LaTeX i løbet af undervisningen så det er vigtigt I har gjort det på forhånd. Det tager ca. 1-2 timer så det er ikke nok at begynde kl. 14 når undervisningen begynder kl. 14:15.

\section{Emneoversigt}
\begin{description}
\item[Uge 1] For nybegyndere kigger vi på basal dokumentopsætning (titel, overskrifter, afsnit, indholdsfortegnelse, lister, simpel matematik), mens for mere øvede brugere kigger vi på avanceret dokumentopsætning (indsætning af kildekode, avanceret matematik, oprettelse af nye kommandoer). 
\item[Uge 2] For nybegyndere ser vi på tabeller, matricer, større opgaver og bibliografi. For mere øvede ser vi på generering af grafik i LaTeX, subfigures, mere avancerede tabeller.
\item[Uge 3] Fremstilling af plakater og slideshow/præsentation i LaTeX.
\end{description}





\end{document}
